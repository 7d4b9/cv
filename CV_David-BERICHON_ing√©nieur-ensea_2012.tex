% Taille de la police du CV et le format
\documentclass[11pt,a4paper, final]{cv}

    % Un certain nombre de packages sont déjà  inclus par la classe CV
    \usepackage[francais]{babel}
    \usepackage[latin1,utf8]{inputenc}
    \usepackage[T2A,T1]{fontenc} 
    \usepackage{csquotes}

    \Language{francais}

    % Méta-données PDF
    \hypersetup{
        pdfauthor   = {David BERICHON},     % C'est vous qui avez fait le PDF
        pdftitle    = {Curriculum Vitae},
    %    pdfsubject  = {},
    %    pdfkeywords = {},
        pdfcreator  = {PDFLaTeX},           % Vous l'avez fait avec LateX !
        pdfproducer = {PDFLaTeX}
    }

    % Couleur des liens hypertextes, si vous voulez les faire ressortir
    \definecolor{urlcolor}{rgb}{0, 0, 0.5}
    
    % Version (0 = simple, devrait rentrer sur une page et 1 = détaillé)
    % Par défaut, tout est inclus dans la version simple sauf les cours (\Course{}).
    % Pour n'inclure un élément que dans la version détaillée, ajouter l'argument [\detailedOnly]
    % (cela n'est toutefois pas implémenté partout, votre nom par exemple ne peut pas ne pas figurer dans la version simple)
    \Detailed{0}
    % Marges
    \geometry{
        hmargin=2.3cm,            % Marges gauche-droite
        vmargin=2.5cm                % Marges haut-bas
    }

    % Largeur de la colonne de gauche
    \setlength{\colwidth}{2.5cm}

    % Espacements
    \smallskipamount=1mm    % Espace entre les différents entrées d'une section
    \medskipamount=3mm        % Espace entre les sections
    \bigskipamount=5mm        % Espace entre le chapeau et l'objectif ou la première section

    \begin{document}

% Le CV commence ici
%%%%%%%%%%%%%%%%%%%%
\begin{center}
    \textbf{INGÉNIEUR ENSEA Promotion 2012}
    \vspace{3mm}
\end{center}

% Chapeau (= informations personnelles)
\begin{heading}

    \Name{David \bsc{Berichon}}
    \Address{
    72, Rue Jean Bleuzen    92~170 Vanves}
    \Telephone{(+33)~6~59~47~08~23}
    \Email{berichon.david@bbox.fr}        % Un lien hypertexte est automatiquement crée pour l'Email
    \Nationality{Française}
    \DateOfBirth{12 février 1984}
    \Age{34}
    \Gender{M}    % M = Masculin, F = Féminin. Est utilisé pour savoir s'il faut écrire Né ou Née par exemple ;-)
    % \MaritalStatus{Célibataire}
    \Mobility{Permis B}
    %\Photo{../logo.jpg}    % Vous pouvez choisir de mettre une photo ou pas.
    %\Photo{photo.JPG}    % Vous pouvez choisir de mettre une photo ou pas.
    % sa dimension est sans importance, au final sa hauteur est fixée à 3cm
    % et son rapport hauteur/largeur est concervé.
    % Meilleure est sa résolution, mieux c'est !

\end{heading}

% Objectif (optionnel)
\begin{objective}
    Ingénieur de Développement Back-End (Golang)
\end{objective}

% Première section
\begin{section}{Expèrience professionnelle}

    \begin{entry}

        \Date{depuis ~~~Mars-2016}
        \Duration{13 mois}
        \Place{\bsc{Ingénieur de développement Back-End GO}}
        \Place{\href{https://oyst.com}{\bsc{Oyst 1-Click}} \bsc{Ingénieur de développement Back-End GO}. Architecture microservice sur AWS. Domaine du paiement}
        \Locality{Paris}
        \Country{France}

        \Activity{- Référent GO et développeur des workers en lien avec le paiement: design logiciel et impléméntation}
        \Activity{- Traitement en microservices GO des flux Kafka/Redis}
        \Activity{- API REST ou gRPC}
        \Activity{- Mise en place workflow et tests d'intégration, BDD en Cucumber}
        \Activity{- Maintiens, support et améliorations continue du code}
        \Activity{- Environment Docker, Github, Aws, Terraform}
        \Activity{- Refactorisation complète de tous les worker Go}
        \Activity{- Bases de donnée Postgres}
        \Activity{- Méthodologie Scrum}

    \end{entry}

    \begin{entry}

        \Date{Sept-2012  Mars-2016}
        \Duration{16 mois}
        \Place{\bsc{Ingénieur Développement C++/Golang}}
        \Place{\href{https://www.ercom.fr}{\bsc{ERCOM-Cryptobox}} \bsc{Ingénieur développeur GO et C++. Domaine de la sécurité :  solution de partage collaboratif avec chiffrement de bout en bout}}
        \Locality{Issy-Les-Moulineaux}
        \Country{France}
        \Activity{- Conception et planification des développements}
        \Activity{- Portage C++11 vers GO et évolution d'un framework backend et d'un SDK multi-plateformes (iOS, Androïd, Windows, Web). En interaction avec une API REST : gestion de sessions utilisateur authentifiées, traitements cryptographiques, implémentation de protocoles propriétaires, synchronisation multi-clients, notifications}
        \Activity{- Génération du code SDK pour Java, Swift, C\# et JavaScript depuis GO : gomobile, jsonrpc, gopherjs}
        \Activity{- Développement d'un CLI sous la forme d'un shell interactif écrit en GO}
        \Activity{- Tests d'intégration avec RobotFramework + Python}
        \Activity{- Gestion de tickets et correction de bugs}
        \Activity{- Environment Docker, Gitlab, JIRA, Confluence, Jenkins, Scrum}
        \Activity{- Méthodologie Scrum}

    \end{entry}

    \begin{entry}
        \Date{Sept-2012  Mai-2015}
        \Duration{29 mois}
        \Place{\href{http://www.vitec.com/}{\bsc{VITEC~MULTIMEDIA}} \bsc{Ingénieur Logiciel} (département software dans le domaine de l'encodage et du streaming vidéo embarqué et temps réel)}
        \Locality{Châtillon-Montrouge}
        \Country{France}
        \Activity{- Conception et développement de logiciels applicatifs de haut niveau, cross-plateforme et multithread C/C++ sous linux}
        \Activity{- Portage du Framework Qt4 sur processeur Intel Atom et développement d'une IHM basée sur Qt}
    \end{entry}

    \begin{entry}

        \Date{2012}
        \Duration{6 mois}
        \Place{\href{http://www.psa-peugeot-citroen.com/}{\bsc{PSA peugeot-citroën}} - R\&D \enquote{DRD/DRIA/MTPE/ERMP/EEL}}
        \Locality{La Garenne Colombe - Paris}
        \Country{France}        
        \Activity{- Mise en place d'un environnement de développement logiciel applicatif pour prototypage rapide d'IHM en C++}
        \Activity{- Personnalisation de systèmes d'exploitation Linux, scripts en BASH}
        \Activity{- Développement d'une IHM en C++ basée sur Qt}
        \Activity{- Documentation technique, formation des usagers, test et validation logiciel de l'IHM}
        \Activity{- Méthodologie Scrum}

    \end{entry}

    \begin{entry}

        \Date{2011}
        \Duration{6 mois}
        \Place{\href{http://www.momagroup.com/}{\bsc{MOMA~(Modélisation, Mesures et Applications)}} - Applications logicielles linux C/C++ : client embarqué (ARM9) et serveur}
        \Locality{Paris VIII}
        \Country{France}

        \Activity{- Modélisation UML}
        \Activity{- Planification, spécifications, développements, mise en recette}
        \Activity{- Documentation}
        \Activity{- Sur processeur ARM9, adaptation noyau linux et device driver pour modems (CPL, GPRS) + thermomètre USB}
        \Activity{- Intégration modules kernel}

    \end{entry}

\end{section}
\newpage
\begin{section}{Formation}

    \begin{entry}

        \Date{2013 (5 days)}
        \Place{\href{http://free-electrons.com/}{\bsc{Free Electrons}}}
        \Locality{Paris (On-site sessions)}
        \Country{France}
        \Activity{\href{http://free-electrons.com/training/kernel/}{\textbf{Embedded Linux kernel and driver development training}}}

        \Course{Configuring, compiling and booting the kernel and its modules}
        \Course{Memory management and accessing hardware - Character device drivers}
        \Course{Processes, scheduling, waiting for resources and interrupt management, Locking}
        \Course{Porting the Linux kernel to a new hardware platform}
        \Course{Working with the community}

    \end{entry}

    \begin{entry}
        \Date{2008 -- 2011}
        \Place{\href{http://www.ensea.fr/}{\bsc{École Nationale Supérieure de l'Électronique et de ses Applications (ENSEA)}}}
        \Locality{Cergy}
        \Country{France}
        \Activity{\textbf{2011 : Spécialisation IS (Informatique et Systèmes)}}

        \Course{Architecture des circuits intégrés numériques}
        \Course{Méthodologie de conception (FPGA, ASIC, CPLD, PLA)}
        \Course{SoC et SoPC}
        \Course{Protocoles et architectures des réseaux}
        \Course{Architectures et processeurs parallèles}
        \Course{Système Unix}
        \Course{Mécanismes internes de Windows}
        \Course{Système de gestion de bases de données}
        \Course{Génie logiciel}
        \Course{Programmation orientée objets - UML}
        \Course{Algorithmique}
    \end{entry}

    \begin{entry}

        \Activity{\textbf{2010 : Spécialisation STC (Signal, Temps réel et Communication)}}

        \Course{Systèmes de Communication}
        \Course{Analyse spectrale et traitement d'antennes}
        \Course{Théorie de la Communication}
        \Course{Traitement du signal temps réel}
        \Course{Traitement numérique du signal}
        \Course{Détection/Estimation}
        \Course{Filtrage adaptatif}
        \Course{Systèmes de Reconnaissance}

    \end{entry}

    \begin{entry}
        \Date{2003 -- 2008}
        \Place{\href{http://www.univ-reunion.fr/}{\bsc{Université de La Réunion}}}
        \Activity{2003 - 2004 : DEUG1 STPI (Sciences et technologies pour l'industrie)}
        \Activity{2004 - 2005 : DEUG1 MIAS (Mathématiques et Informatique Appliqués aux Sciences)}
        \Activity{2005 - 2008 : LICENCE PHYSIQUE}
    \end{entry} 

\end{section}

\begin{section}{Diplômes et langues}

    \singleEntry{$\cdot$ \bsc{Free Electron} : Embedded Linux kernel and driver development}
    \singleEntry{$\cdot$ Ingénieur ENSEA}
    \singleEntry{$\cdot$ DEUG Physique}
    \singleEntry{$\cdot$ Baccalauréat série S spécialité physique}
    
    \smallSkip

    \begin{entry}
        \Skill{Français~:}
        \Activity{Langue maternelle}
    \end{entry}

    \begin{entry}
        \Skill{Anglais~:}
        \Activity{Parlé et écrit courant}
        \Activity{Diplôme du \href{http://www.fr.toeic.eu/index.php?id=2761}{TOEIC}}
    \end{entry}

\end{section}

\begin{section}{Loisirs et activités}
    \singleEntry{$\cdot$ Sports : Trail/Course à pied (Marathon De Paris, Saintélyon, 6000D), natation}
    \singleEntry{$\cdot$ Cinema, musique : cor d'harmonie, niveau 3eme cycle de conservatoire}
    \singleEntry{$\cdot$ Informatique, internet, nouvelles technologies}
\end{section}

% Le CV termine ici
%%%%%%%%%%%%%%%%%%%

\end{document}
